%%=============================================================================
%% Inleiding
%%=============================================================================

\chapter{\IfLanguageName{dutch}{Inleiding}{Introduction}}
\label{ch:inleiding}

%De inleiding moet de lezer net genoeg informatie verschaffen om het onderwerp
%te begrijpen en in te zien waarom de onderzoeksvraag de moeite waard is om te
%onderzoeken. In de inleiding ga je literatuurverwijzingen beperken, zodat de
%tekst vlot leesbaar blijft. Je kan de inleiding verder onderverdelen in secties
%als dit de tekst verduidelijkt. Zaken die aan bod kunnen komen in de
%inleiding~\autocite{Pollefliet2011}:
%
%\begin{itemize}
%  \item context, achtergrond
%  \item afbakenen van het onderwerp
%  \item verantwoording van het onderwerp, methodologie
%  \item probleemstelling
%  \item onderzoeksdoelstelling
%  \item onderzoeksvraag
%  \item \ldots
%\end{itemize}

\section{Achtergrond}
\label{sec:achtergrond}

Mobiele applicaties zijn niet meer weg te denken uit het dagelijkse leven. Een
app die je wakker maakt, één die je vertelt welke plannen je hebt die dag, een
andere om je op de hoogte te brengen van de gebeurtenissen in de wereld en ga zo
maar door. Al deze apps moeten natuurlijk voortdurend onderhouden en verbeterd
worden door de ontwikkelaars, terwijl deze zich tegelijk ook bezighouden met
het ontwikkelen van nieuwe applicaties. Eén van de grootste uitdagingen op het
vlak van mobiele applicatieontwikkeling is dat er meerdere besturingssystemen
bestaan voor mobiele apparaten (waarvan Android en iOS zonder twijfel de
populairste zijn). Ontwikkelaars willen natuurlijk dat zoveel mogelijk mensen
gebruik kunnen maken van hun applicatie: ze moeten dus zorgen dat de applicatie
werkt op alle besturingssystemen. De app telkens opnieuw gaan schrijven voor elk
systeem is echter een tijdrovende en repetitieve bezigheid. Om dit probleem uit
de wereld te helpen werden cross-platform frameworks in het leven geroepen:
frameworks die de ontwikkelaar in staat stellen om met één enkele broncode de
applicatie op alle systemen te laten werken.

De precieze voordelen (en uiteraard ook nadelen) van cross-platform frameworks
worden in de studie uitgebreid besproken. Men kan wel direct intuïtief aanvoelen
dat ze een grote tijdswinst opleveren en dus steeds populairder worden. Doordat
de populariteit ervan stijgt zijn ook steeds meer producenten van frameworks
zich hierop gaan storten. Er zijn de laatste jaren zeer veel verschillende
frameworks uitgebracht, elk met hun eigen sterktes en zwaktes. De vraag dringt
zich dan ook op welk framework de beste keuze is om mee aan de slag te gaan als
team van ontwikkelaars in de komende jaren. 

\section{\IfLanguageName{dutch}{Probleemstelling}{Problem Statement}}
\label{sec:probleemstelling}

%Uit je probleemstelling moet duidelijk zijn dat je onderzoek een meerwaarde
%heeft voor een concrete doelgroep. De doelgroep moet goed gedefinieerd en
%afgelijnd zijn. Doelgroepen als ``bedrijven,'' ``KMO's,'' systeembeheerders,
%enz.~zijn nog te vaag. Als je een lijstje kan maken van de
%personen/organisaties die een meerwaarde zullen vinden in deze bachelorproef
%(dit is eigenlijk je steekproefkader), dan is dat een indicatie dat de
%doelgroep goed gedefinieerd is. Dit kan een enkel bedrijf zijn of zelfs één
%persoon (je co-promotor/opdrachtgever).

Bij het ontwikkelen van mobiele applicaties zijn er vele zaken waar rekening mee gehouden moet worden, zowel vanuit het standpunt van de ontwikkelaar als dat van de gebruikers. Voor de ontwikkelaar is het onder andere belangrijk om slechts één keer code te moeten schrijven, dat de code onderhoudbaar is en dat de applicatie veilig is om te gebruiken. Vanuit het standpunt van de gebruikers is het belangrijk om een applicatie zonder fouten te hebben die geen last heeft van performantieproblemen, die er mooi uitziet en die regelmatig updates ontvangt zodat deze steeds voldoet aan de laatste nieuwe normen. 

Op de B2B markt is het tevens heel erg belangrijk dat een applicatie werkt op alle besturingssystemen: indien ze niet allemaal beschikken over hetzelfde besturingssysteem moeten ze natuurlijk nog steeds in staat zijn om de applicaties van het bedrijf te gebruiken. Om aan deze heel belangrijke eis te kunnen voldoen is het gebruik van een cross-platform een absolute meerwaarde: slechts één keer schrijven, een applicatie die op alle systemen beschikbaar is en die er overal hetzelfde uitziet. 

Zoals reeds eerder vermeld is er echter een zeer groot aanbod van dergelijke frameworks, elk met hun eigen sterktes en zwaktes. Beslissen met welk framework er de komende jaren gewerkt zal worden is dus allesbehalve een eenvoudige opgave, vandaar dit uitgebreide onderzoek.

\section{\IfLanguageName{dutch}{Onderzoeksvraag}{Research question}}
\label{sec:onderzoeksvraag}

%Wees zo concreet mogelijk bij het formuleren van je onderzoeksvraag. Een
%onderzoeksvraag is trouwens iets waar nog niemand op dit moment een antwoord
%heeft (voor zover je kan nagaan). Het opzoeken van bestaande informatie (bv.
%``welke tools bestaan er voor deze toepassing?'') is dus geen onderzoeksvraag.
%Je kan de onderzoeksvraag verder specifiëren in deelvragen. Bv.~als je
%onderzoek gaat over performantiemetingen, dan 

\subsection{Hoofdonderzoeksvraag}

Zoals reeds eerder aangegeven zal dit onderzoek zich richten op het vergelijken van cross-platform frameworks. Hieruit volgt dan ook de volgende onderzoeksvraag:

Welk cross-platform framework is de beste keuze om de komende jaren mee aan de slag te gaan voor het ontwikkelen van mobiele applicaties bestemd voor de B2B-markt?

In de conclusie zal een antwoord gegeven worden op deze onderzoeksvraag.

\subsection{Deelonderzoeksvragen}

Om tot een antwoord te kunnen komen op de hoofdonderzoeksvraag wordt in deze studie ook een antwoord gezocht op de volgende deelonderzoeksvragen:

\begin{itemize}
    \item Wat is een cross-platform framework en wat zijn de sterktes en zwaktes hiervan?
    \item Wat zijn de voor- en nadelen van React Native en Flutter?
    \item Wat zijn de verschillen en gelijkenissen tussen React Native en Flutter?
\end{itemize}

In de loop van dit onderzoek wordt een antwoord gegeven op elk van deze deelonderzoeksvragen. In de conclusie van dit onderzoek wordt er nog eens een samenvatting gegeven van deze antwoorden.

\section{\IfLanguageName{dutch}{Onderzoeksdoelstelling}{Research objective}}
\label{sec:onderzoeksdoelstelling}

%Wat is het beoogde resultaat van je bachelorproef? Wat zijn de criteria voor
%succes? Beschrijf die zo concreet mogelijk. Gaat het bv. om een
%proof-of-concept, een prototype, een verslag met aanbevelingen, een
%vergelijkende studie, enz.

Het hoofddoel van deze studie is om een cross-platform framework te kunnen aanraden om de komende jaren op in te zetten voor het ontwikkelen van B2B-applicaties (dit is dus een antwoord kunnen geven op de hoofdonderzoeksvraag). Een tweede doel is lezers van dit onderzoek te informeren over de verschillen en gelijkenissen tussen React Native en Flutter. Dit onderzoek is geslaagd indien lezers hun eigen geïnformeerde keuze kunnen maken over welk framework voor hen de beste de keuze is.

\section{\IfLanguageName{dutch}{Hypothese}{Hypothesis}}
\label{sec:hypothese}

Voor er van start gegaan wordt met dit onderzoek wordt er een hypothese opgesteld, gestoeld op de reeds gevonden informatie tijdens het schrijven van het onderzoeksvoorstel, te vinden in bijlage A. Een eerste hypothese is dat React Native op dit moment de beste keuze is voor het ontwikkelen van mobiele applicaties, maar dat andere frameworks deze positie in de toekomst zeker kunnen overnemen. Een tweede hypothese is dat er geen eenduidig antwoord bestaat op de onderzoeksvraag, aangezien de keuze zeer sterk zal afhangen van de reeds bestaande kennis en voorkeuren van een team ontwikkelaars.

\section{\IfLanguageName{dutch}{Opzet van deze bachelorproef}{Structure of this
        bachelor thesis}}
\label{sec:opzet-bachelorproef}

% Het is gebruikelijk aan het einde van de inleiding een overzicht te
% geven van de opbouw van de rest van de tekst. Deze sectie bevat al een aanzet
% die je kan aanvullen/aanpassen in functie van je eigen tekst.

De rest van deze bachelorproef is als volgt opgebouwd:

In hoofdstuk~\ref{ch:stand-van-zaken} wordt een overzicht gegeven van de stand
van zaken binnen het domein van cross-platform frameworks op basis van een literatuurstudie.

Daarna volgt in hoofdstuk~\ref{ch:methodologie} het uiteindelijke onderzoek om een antwoord te kunnen geven op de verschillende onderzoeksvragen.

% TODO: Vul hier aan voor je eigen hoofstukken, één of twee zinnen per hoofdstuk

Tot slot wordt er in het hoofdstuk \ref{ch:conclusie} een samenvatting van de conclusies van deze studie gegeven. Het antwoord op elk van de onderzoeksvragen wordt kort samengevat en de aanzet voor toekomstig onderzoek in dit vakgebied wordt gegeven.



