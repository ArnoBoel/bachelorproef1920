%%=============================================================================
%% Inleiding
%%=============================================================================

\chapter{\IfLanguageName{dutch}{Inleiding}{Introduction}}
\label{ch:inleiding}

%De inleiding moet de lezer net genoeg informatie verschaffen om het onderwerp te begrijpen en in te zien waarom de onderzoeksvraag de moeite waard is om te onderzoeken. In de inleiding ga je literatuurverwijzingen beperken, zodat de tekst vlot leesbaar blijft. Je kan de inleiding verder onderverdelen in secties als dit de tekst verduidelijkt. Zaken die aan bod kunnen komen in de inleiding~\autocite{Pollefliet2011}:
%
%\begin{itemize}
%  \item context, achtergrond
%  \item afbakenen van het onderwerp
%  \item verantwoording van het onderwerp, methodologie
%  \item probleemstelling
%  \item onderzoeksdoelstelling
%  \item onderzoeksvraag
%  \item \ldots
%\end{itemize}

\section{Achtergrond}
\label{sec:achtergrond}

Chatbots worden steeds vaker en vaker gebruikt als eeste aanspreekpunt op een website, waardoor meer gebruikers geholpen kunnen worden met hun specifieke problemen of vragen. Ze werken volledig autonoom en moeten dus voldoen aan bepaalde kwaliteitseisen, zodat het imago van het berdijf niet geschonden wordt door slechte klantenservice. Als een klant bijvoorbeeld een bepaalde vraag stelt wordt er verwacht van de chatbot dat deze steeds het correcte (en dus hetzelfde) antwoord zal geven. 

Net zoals bij andere softwareprojecten worden steeds nieuwe features toegevoegd aan een chatbot, zodat deze steeds beter en beter worden. Als ontwikkelaar is het onontbeerlijk om er zeker van te kunnen zijn dat een nieuwe feature die wordt toegevoegd geen andere reeds bestaande features kapot maakt. Om dit te kunnen vaststellen wordt aan regressietesten gedaan: testen van de onderdelen van de applicatie die niet veranderd zijn om zich er van te vergewissen dat deze nog steeds werken.

\section{\IfLanguageName{dutch}{Probleemstelling}{Problem Statement}}
\label{sec:probleemstelling}

%Uit je probleemstelling moet duidelijk zijn dat je onderzoek een meerwaarde heeft voor een concrete doelgroep. De doelgroep moet goed gedefinieerd en afgelijnd zijn. Doelgroepen als ``bedrijven,'' ``KMO's,'' systeembeheerders, enz.~zijn nog te vaag. Als je een lijstje kan maken van de personen/organisaties die een meerwaarde zullen vinden in deze bachelorproef (dit is eigenlijk je steekproefkader), dan is dat een indicatie dat de doelgroep goed gedefinieerd is. Dit kan een enkel bedrijf zijn of zelfs één persoon (je co-promotor/opdrachtgever).

Testen van software is een proces dat overal reeds goed ingeburgerd is (of zou moeten zijn). In de wereld van de chatbot ontwikkeling is dit niet anders. Het is echter belangrijk dat features niet enkel getest worden op het moment dat ze worden geïmplenteerd, maar ook als andere features worden toegevoegd. Op deze manier kan men zeker zijn dat reeds ontwikkelde features nog steeds werken zoals verwacht. Dit onderzoek zal een aanbeveling doen aan de ontwikkelaars van chatbots binnen Zoovu over hoe het regressie testen van chatbots kan aangepakt worden. Dit houdt een grondig onderzoek in naar de reeds bestaande technieken om een chatbot te testen en hoe deze gebruikt kunnen worden om op een efficiënte manier aan regressietesten te gaan doen. 

\section{\IfLanguageName{dutch}{Onderzoeksvraag}{Research question}}
\label{sec:onderzoeksvraag}

%Wees zo concreet mogelijk bij het formuleren van je onderzoeksvraag. Een onderzoeksvraag is trouwens iets waar nog niemand op dit moment een antwoord heeft (voor zover je kan nagaan). Het opzoeken van bestaande informatie (bv. ``welke tools bestaan er voor deze toepassing?'') is dus geen onderzoeksvraag. Je kan de onderzoeksvraag verder specifiëren in deelvragen. Bv.~als je onderzoek gaat over performantiemetingen, dan 

Hoe kunnen chatbot ontwikkelaars binnen Zoovu efficiënt aan regressietesten doen om de kwaliteit van reeds bestaande onderdelen te waarborgen na het toevoegen van een nieuw onderdeel?

\section{\IfLanguageName{dutch}{Onderzoeksdoelstelling}{Research objective}}
\label{sec:onderzoeksdoelstelling}

%Wat is het beoogde resultaat van je bachelorproef? Wat zijn de criteria voor succes? Beschrijf die zo concreet mogelijk. Gaat het bv. om een proof-of-concept, een prototype, een verslag met aanbevelingen, een vergelijkende studie, enz.

Het beoogde resultaat van deze bachelorproef is om een verslag af te leveren met een aanbeveling hoe het regressietesten van chatbots best aangepakt kan worden en met welke technologieën. 

\section{\IfLanguageName{dutch}{Opzet van deze bachelorproef}{Structure of this bachelor thesis}}
\label{sec:opzet-bachelorproef}

% Het is gebruikelijk aan het einde van de inleiding een overzicht te
% geven van de opbouw van de rest van de tekst. Deze sectie bevat al een aanzet
% die je kan aanvullen/aanpassen in functie van je eigen tekst.

De rest van deze bachelorproef is als volgt opgebouwd:

In Hoofdstuk~\ref{ch:stand-van-zaken} wordt een overzicht gegeven van de stand van zaken binnen het testen van chatbots op basis van een literatuurstudie.

In Hoofdstuk~\ref{ch:methodologie} wordt de methodologie toegelicht en worden de gebruikte onderzoekstechnieken besproken om een antwoord te kunnen formuleren op de onderzoeksvraag.

% TODO: Vul hier aan voor je eigen hoofstukken, één of twee zinnen per hoofdstuk

In Hoofdstuk~\ref{ch:conclusie}, tenslotte, wordt de conclusie gegeven en een antwoord geformuleerd op de onderzoeksvraag. Daarbij wordt ook een aanzet gegeven voor toekomstig onderzoek binnen dit domein.


