%%=============================================================================
%% Samenvatting
%%=============================================================================

% TODO: De "abstract" of samenvatting is een kernachtige (~ 1 blz. voor een
% thesis) synthese van het document.
%
% Deze aspecten moeten zeker aan bod komen:
% - Context: waarom is dit werk belangrijk?
% - Nood: waarom moest dit onderzocht worden?
% - Taak: wat heb je precies gedaan?
% - Object: wat staat in dit document geschreven?
% - Resultaat: wat was het resultaat?
% - Conclusie: wat is/zijn de belangrijkste conclusie(s)?
% - Perspectief: blijven er nog vragen open die in de toekomst nog kunnen
%    onderzocht worden? Wat is een mogelijk vervolg voor jouw onderzoek?
%
% LET OP! Een samenvatting is GEEN voorwoord!

%%---------- Nederlandse samenvatting -----------------------------------------
%
% TODO: Als je je bachelorproef in het Engels schrijft, moet je eerst een
% Nederlandse samenvatting invoegen. Haal daarvoor onderstaande code uit
% commentaar.
% Wie zijn bachelorproef in het Nederlands schrijft, kan dit negeren, de inhoud
% wordt niet in het document ingevoegd.

\IfLanguageName{english}{%
\selectlanguage{dutch}
\chapter*{Samenvatting}
\selectlanguage{english}
}{}

%%---------- Samenvatting -----------------------------------------------------
% De samenvatting in de hoofdtaal van het document

\chapter*{\IfLanguageName{dutch}{Samenvatting}{Abstract}}

De hedendaagse wereld draait steeds meer en meer rond technologie. Mensen shoppen steeds vaker bij webshops, zoeken op websites naar bepaade informatie of willen een klantendienst online bereiken. Een service die vele bedrijven daarom aanbieden is de mogelijkheid om op hun website/webshop te kunnen chatten met een medewerker die hun specifieke vragen kan beantwoorden. Om de grote toevloed aan chatberichten de baas te kunnen kiezen vele bedrijven voor een chatbot als eerste aanpsreekpunt voor een klant. Deze kan reeds veel informatie bekomen door met de klant in gesprek te gaan. Een chatbot werkt volledig automatisch en ontlast zodoende de medewerkers van het bedrijf. 

Aangezien een chatbot volledig autonoom moet kunnen werken is het noodzakelijk om de kwaliteit van de chatbot te kunnen garanderen. Om zich er van te vergewissen dat de chatbot aan de eisen voldoet en zinvolle antwoorden geeft op de input van gebruikers maken ontwikkelaars gebruik van testen. Deze testen worden geschreven op hetzelfde moment als dat de code voor die bepaalde feature geschreven wordt. Op dat moment kan er dus met zekerheid gezegd worden dat die bepaalde feature voldoet aan de kwaliteitseisen die gesteld worden en kan de chatbot opgeleverd worden met deze nieuwe feature. 

Zoals in elk software project komen er ook bij een chatbot voortdurend nieuwe features bij. Dit kan bijvoorbeeld gaan over het toevoegen van een nieuwe taal die ondersteund wordt, bepaalde onderwerpen toevoegen die de chatbot verder kan beantwoorden, extra taken opnemen dan louter informatie geven, ... Er zijn enorm veel mogelijkheden om uit te breiden. Al deze uitbreidingen moeten ook op hun beurt getest worden en mogen andere reeds werkende features natuurlijk niet kapot gaan maken. Dit is waar regressie testen aan bod komt: het uitvoeren van testen van eerder geschreven features zodat deze nog steeds werken na het toevoegen van een nieuwe feature. Op deze manier kan een ontwikkelaar er zeker van zijn dat een gebruiker dezelfde service als ervoor zal kunnen ervaren.

% Taak

%