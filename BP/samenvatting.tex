%%=============================================================================
%% Samenvatting
%%=============================================================================

% TODO: De "abstract" of samenvatting is een kernachtige (~ 1 blz. voor een
% thesis) synthese van het document.
%
% Deze aspecten moeten zeker aan bod komen:
% - Context: waarom is dit werk belangrijk?
% - Nood: waarom moest dit onderzocht worden?
% - Taak: wat heb je precies gedaan?
% - Object: wat staat in dit document geschreven?
% - Resultaat: wat was het resultaat?
% - Conclusie: wat is/zijn de belangrijkste conclusie(s)?
% - Perspectief: blijven er nog vragen open die in de toekomst nog kunnen
%    onderzocht worden? Wat is een mogelijk vervolg voor jouw onderzoek?
%
% LET OP! Een samenvatting is GEEN voorwoord!

%%---------- Nederlandse samenvatting -----------------------------------------
%
% TODO: Als je je bachelorproef in het Engels schrijft, moet je eerst een
% Nederlandse samenvatting invoegen. Haal daarvoor onderstaande code uit
% commentaar.
% Wie zijn bachelorproef in het Nederlands schrijft, kan dit negeren, de inhoud
% wordt niet in het document ingevoegd.

\IfLanguageName{english}{%
\selectlanguage{dutch}
\chapter*{Samenvatting}
\selectlanguage{english}
}{}

%%---------- Samenvatting -----------------------------------------------------
% De samenvatting in de hoofdtaal van het document

\chapter*{\IfLanguageName{dutch}{Samenvatting}{Abstract}}

In de wereld van vandaag beschikt bijna iedereen over een smartphone. Om deze apparaten ten volle te kunnen benutten zijn er miljoenen applicaties beschikbaar die op hun beurt kunnen rekenen op miljoenen gebruikers. Om aan de steeds groter wordende verwachtingen van de gebruikers te voldoen spenderen ontwikkelaars zeer veel tijd aan het bouwen van nieuwe applicaties en het onderhouden en verbeteren van reeds bestaande applicaties. Bij het ontwikkelen van mobiele applicaties komen ontwikkelaars steeds weer dezelfde uitdaging tegen: er zijn verschillende populaire besturingssystemen waarop deze applicaties moeten kunnen werken voor mobiele gebruikers, zijnde iOS, Android en Windows (in de B2B markt). Aangezien ontwikkelaars natuurlijk niet op voorhand reeds een heel groot percentage potentiële gebruikers willen uitsluiten moeten alle applicaties ontwikkeld worden voor alle besturingssystemen. Dit is een tijdrovend en repetitief proces, aangezien er in se telkens hetzelfde gemaakt wordt. Om dit probleem aan te pakken zijn cross-platform frameworks in het leven geroepen. Dit zijn frameworks die de ontwikkelaars in staat stellen om met één enkele broncode hun applicatie te laten werken op verschillende besturingssystemen. Dit levert een grote tijdswinst op voor de ontwikkelaars en is dus een zeer interessante technologie om mee aan de slag te gaan. De laatste jaren zijn meer en meer producenten van frameworks zich hier dan ook op gaan focussen met als gevolg dat er nu een brede keuze is aan verschillende cross-platform frameworks. De vraag dringt zich dan ook op welke van deze bestaande frameworks de beste keuze is om in de toekomst mee aan de slag te gaan voor het ontwikkelen van mobiele applicaties.

Om een objectief antwoord te kunnen geven op deze vraag wordt er eerst een lijst met eisen opgesteld waar de frameworks zeker aan moeten voldoen. In deze studie zullen drie cross-platform frameworks die aan alle eisen voldoen met elkaar vergeleken worden: React Native (Facebook), Flutter (Google) en .NET MAUI (Microsoft). Concreet zal er gewerkt worden rond de volgende onderzoeksvraag: welk cross platform framework is de beste keuze voor een team van ontwikkelaars om de komende jaren op in te zetten voor het ontwikkelen van mobiele applicaties. Hierbij zal gekeken worden naar de gebruiksvriendelijkheid voor zowel de ontwikkelaar als de gebruiker, de ondersteuning van het framework (community, libraries, documentatie), de snelheid van apps ontwikkeld met dit framework en de mogelijkheid om te communiceren met de native API's. Van elke technologie zullen de sterke en zwakke punten opgesomd worden, ze zullen onderling vergeleken worden op de hiervoor benoemde punten en in de conclusie zal een aanbeveling gedaan worden welk framework het meest 'future-proof' is. 

% Taak

%