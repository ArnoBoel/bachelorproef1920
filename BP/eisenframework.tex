\chapter{\IfLanguageName{dutch}{Eisen framework}{Framework requirements}}
\label{ch:eisen-framework}

Alvorens een vergelijking kan gemaakt worden tussen enkele frameworks moet er bepaald worden welke frameworks voldoen aan de eisen die gesteld worden door het team. De eisen worden onderverdeeld in functionele en niet-functionele eisen. De eisen waaraan de frameworks binnen deze studie moeten voldoen zijn te vinden in de onderstaande secties. In een volgend hoofdstuk worden vervolgens alle cross-platform frameworks van de opgestelde longlist afgetoetst aan deze eisen. Enkel frameworks die voldoen aan al deze eisen komen in aanmerking om mee genomen te worden in de vergelijkende studie.

\section{Functionele eisen}
\label{sec:functioneleEisen}

Eerst volgt voor de duidelijkheid een opsomming van de functionele eisen. Vervolgens wordt er meer uitleg gegeven waarom deze eisen gesteld worden.

\begin{enumerate}
    \item Apps zien er native uit
    \item Apps hebben dezelfde performantie als native apps
    \item Apps kunnen offline gebruikt worden (indien er functies zijn die geen netwerkverbinding nodig hebben)
    \item Er kan gebruik gemaakt worden van de native API's (native functionaliteiten)
\end{enumerate}

Voor gebruikers van een applicatie is het zeer belangrijk dat de gebruikersinterface van een applicatie overeenkomt met wat ze gewoon zijn van andere applicaties. Dit houdt in dat ze op dezelfde manier moeten kunnen navigeren, knoppen en symbolen moeten er hetzelfde kunnen uitzien, widgets moeten native aanvoelen en ga zo maar door. Dit alles zorgt ervoor dat het gebruiksgemak van de applicatie gevoelig beter wordt. Niet enkel hoe applicaties er uit zien is belangrijk voor gebruikers, ook de performantie ervan is een zeer belangrijk punt. Niemand heeft graag een app die hapert of ellendig lang duurt om te laden. 

Verder is het ook belangrijk dat een app die beschikt over functionaliteit waarvoor geen netwerkconnectie vereist is ook offline gebruikt kan worden. De applicatie moet dus op het apparaat zelf geïnstalleerd kunnen worden en er moet lokale data bijgehouden kunnen worden. Een laatste eis is er één vanuit het oogpunt van de ontwikkelaars: om de gebruikers een native ervaring te kunnen aanbieden heeft de ontwikkelaar toegang nodig tot de native API's. Door deze eis te stellen wordt er verzekerd dat het framework toelaat om alle aanwezige hardwarefunctionaliteiten van een systeem te gaan benutten.

\section{Niet-functionele eisen}
\label{sec:nietFunctioneleEisen}

Net als bij de functionele eisen volgt ook hier eerst een opsomming, waarna de motivatie voor het stellen van deze eisen gegeven wordt.

\begin{enumerate}
    \item Reeds aanwezige kennis van de programmeertaal binnen het team
    \item Grote en actieve community (en dus een populair framework)
    \item Voldoende documentatie beschikbaar
    \item Regelmatige updates van het framework
    \item Recente laatste stabiele versie (geen verouderd framework)
    \item Toekomstige ondersteuning gegarandeerd    
\end{enumerate}

Om vlot aan de slag te kunnen met het framework is het een grote meerwaarde als de programmeertaal reeds gekend is binnen het team. Op deze manier kan het team zich onmiddellijk focussen op de specifieke eigenschappen van het framework, in plaats van eerst de syntax van de taal te moeten leren. Eens de ontwikkelaars effectief een app aan het schrijven zijn is de kans zeer groot dat ze zaken tegen komen die ze niet direct zelf kunnen oplossen. In deze situaties is het noodzakelijk om te kunnen terugvallen op de officiële documentatie van het framework en op een community die klaar staat om te helpen (denk maar aan issues op GitHub, vragen op Stackoverflow, ...). Verder is een actieve community een goede indicatie dat het framework 'leeft' en dus ook updates zal ontvangen. Regelmatige updates zijn noodzakelijk voor elk softwareprogramma. Er zijn voortdurend nieuwe ontwikkelingen en nieuwe bedreigingen. Om moderne en veilige applicaties te schrijven (en deze zo te houden) is een framework met regelmatige updates dus onontbeerlijk. Vanuit datzelfde oogpunt is ook een recente stabiele versie een belangrijke eis. Tot slot is het doel van deze studie om een aanbeveling te kunnen doen van een framework dat geschikt is om de komende jaren mee aan de slag te gaan. Het is dus zeer belangrijk dat toekomstige ondersteuning gegarandeerd is. Hiermee wordt bedoeld dat er updates zullen uitgebracht blijven worden, met nieuwe functionaliteiten en de laatste nieuwe beveiligingsmethodes.