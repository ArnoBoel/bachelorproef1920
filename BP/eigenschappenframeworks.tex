\chapter{\IfLanguageName{dutch}{Eigenschappen gekozen frameworks}{Features of chosen frameworks}}
\label{ch:eigenschappen-frameworks}

In het vorige hoofdstuk werd aan de hand van een ondervraging \autocite{Liu2020} en de gestelde eisen voor een framework een lijst van frameworks samengesteld die in het vervolg van deze studie met elkaar vergeleken zullen worden. In dit hoofdstuk worden deze frameworks bechreven, elk met hun algemene achtergrond, sterke punten en minder sterke punten zodat de lezer een goed idee krijgt van de afzonderlijke frameworks. In het volgende hoofdstuk zullen deze vier frameworks onderling vergeleken worden op de punten die zijn vastgelegd in sectie ???

\section{React Native}
\label{sec:detailsReactNative}

In sectie \ref{subsec:ReactNative} werd reeds kort de algemene achtergrond van React Native besproken. Er werd reeds gesteld dat het framework steunt op React en dus een groot voordeel oplevert voor ontwikkelaars die reeds vertrouwd zijn met React. Dit is natuurlijk niet het enige voordeel van React Native. In de volgende paragrafen worden de sterke punten van React Native verder besproken.

Zoals in de vorige pragraaf reeds gesteld werd is het feit dat het framework gebaseerd is op React een groot voordeel voor ontwikkelaars die reeds React kennen (de taal en manier van werken is reeds gekend). Ook voor ontwikkelaars die nog geen achtergrond hebben in React levert dit echter een groot voordeel op: ook de 'hot reload' van React werd meegenomen in het framework. Dit houdt in dat een ontwikkelaar aanpassingen kan doen, deze opslaan en direct het resultaat van deze wijzigingen kan zien. Dit in tegenstelling tot native ontwikkeling, waar de build van een applicatie al gauw meerdere minuten in beslag neemt. Voor het op punt stellen van de UI van de applicatie levert de 'hot reload' dus een gevoelige tijdsbesparing op.

React Native heeft natuurlijk ook enkele sterke punten die niks te maken hebben met het feit dat het gebaseerd is op React. Eén daarvan is de bescrhijving van UI componenten met eigen componenten binnen het framework. Dit houdt in dat het framework beschikt over een algemene beschrijving voor een component en bij het effectieve deployen van de applicatie op een bepaald platform wordt deze algemene beschrijving gelinkt aan de specifieke component van dat platform. Het grote voordeel hiervan is dat de ontwikkelaar geen enkele kennis nodig heeft van de native UI componenten. Kennis van de beschrijving van React Native van componenten is voldoende om UI's te maken die er native uitzien. 

Applicaties die er native uitzien ondanks het feit dat ze cross-platform geschreven zijn is één van de grootste voordelen van React Native. Met een eigen definitie van componenten zoals besproken in de vorige paragraaf en directe toegang tot de native API's kunnen ontwikkelaars dezelfde kwaliteitsvolle UI's leveren aan de gebruikers van de applicatie als met een native applicatie. Gezien de overvloed van bestaande applicaties en de eisen die gebruikers stellen aan het uitzicht van een applicatie is dit absoluut noodzakelijk om een applicatie te schrijven die met plezier gebruikt wordt.

Verder is React Native een zeer populair framework (zie figuur \ref{fig:frameworkPopularity}). Het is ontwikkeld en wordt onderhouden door Facebook, maar het framework is open source en alle code ervan is integraal te vinden op GitHub. Door de grote populariteit en het open source karakter van het framework zijn er zeer veel contributors aan het framework en zijn er dus ook regelmatige updates en bug fixes. De contributors van het framework zijn bovendien niet enkel individuen, ook enkele andere bedrijven (zoals o.a. Microsoft) helpen mee met de ontwikkeling van React Native \autocite{Nakazawa2019}. Door de grote en actieve communicty zijn er dus regelmatige updates en bugfixes, maar dat is niet het enige voordeel. Er zijn ook zeer veel externe libraries beschikbaar die veel voorkomende functionaliteiten omvatten in simpele methodes, zodat de ontwikkelaar niet alles zelf hoeft te schrijven. Dit levert een grote tijdswinst op en kan er ook voor zorgen dat er bepaalde functionaliteit toegevoegd kan worden aan de app die de ontwikkelaar zelf niet had kunnen schrijven.


\section{Flutter}
\label{detailsFlutter}

In subsectie \ref{subsec:Flutter} werd reeds gesteld dat Flutter een UI toolkit is die ontwikkeld is door Google en dat er zeer recent nog een nieuwe stabiele release is uitgebracht. Het is dus een framework dat de gebruiker toegang geeft tot de laatste nieuwe eigenschappen van de verschillende platformen.

Om de ontwikkelaar in staat te stellen om op een eenduidige manier de gebruikersinterface van de applicatie vast te leggen maakt Flutter gebruik van widgets. Dit zijn componenten die door het framework vastgelegd zijn en gelinkt zijn aan de native voorstelling van deze componenten. Op deze manier hoeft de ontwikkelaar dus geen kennis te hebben van de native componenten om deze te kunnen gebruiken in de applicatie. 

Een ander voordeel van te werken met widgets is dat de performantie van de gebruikerinterface gelijk is aan die van een native applicatie, aangezien de componenten worden omgezet naar de native componenten. De gebruikers van een applicatie kunnen dus rekenen op een applicatie die er niet alleen uitziet maar ook presteert als een native applicatie.

Tijdens het ontwikkelen van de gebruikersinterface van een applicatie is het belangrijk om de aanpassingen die de ontwikkelaar doet visueel te kunnen raadplegen om het effect van deze aanpassingen te zien. Flutter biedt op dit vlak een zeer groot voordeel in vergelijking met native ontwikkeling: waar het builden van een native applicatie al snel enkele minuten in beslag neemt beschikt Flutter over een 'hot reload'. Dit zorgt ervoor dat aanpassingen aan de applicatie kunnen gebeuren zonder dat de applicatie herstart moet worden. Het grote voordeel hiervan is dat de ontwikkelaar verder kan gaan met testen van hetzelfde punt als voor de aanpassingen. De hot reload functie van Flutter gaat echter verder dan enkel aanpassingen aan de gebruikersinterface kunnen tonen met een hot reload. Ook aanpassingen aan functies en bug fixes kunnen met een hot reload toegepast worden. Op deze manier is het dus makkelijker om een bug te proberen reproduceren om te gaan kijken of de bugfix effectief gewerkt heeft. Het is dus duidelijk dat de hot reload functie van Flutter een grote tijdswinst oplevert voor de ontwikkelaar!

Een laatste grote voordeel van Flutter werd reeds besproken in subsectie \ref{subsec:Flutter}: de taal waarin Flutter geschreven wordt (Dart) kan rechtstreeks gecompileerd worden naar native machine code. De applicatie zal op de verschillende platformen zich dus kunnen gedragen als een native applicatie: de gebruiker beschikt dus over een applicatie met de perfomantie en gebruikersinterface van een native applicatie, terwijl de ontwikkelaar beschikt over voordelen van een cross-platform framework!
