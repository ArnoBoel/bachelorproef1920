\chapter{\IfLanguageName{dutch}{Eigenschappen gekozen frameworks}{Features of chosen frameworks}}
\label{ch:eigenschappen-frameworks}

In het vorige hoofdstuk werd aan de hand van een ondervraging \autocite{Liu2020} en de gestelde eisen voor een framework een lijst van frameworks samengesteld die in het vervolg van deze studie met elkaar vergeleken zullen worden. In dit hoofdstuk worden deze frameworks bechreven, elk met hun algemene achtergrond, sterke punten en minder sterke punten zodat de lezer een goed idee krijgt van de afzonderlijke frameworks. In het volgende hoofdstuk zullen deze vier frameworks onderling vergeleken worden op de punten die zijn vastgelegd in sectie ???

\section{React Native}
\label{sec:detailsReactNative}

In sectie \ref{subsec:ReactNative} werd reeds kort de algemene achtergrond van React Native besproken. Er werd reeds gesteld dat het framework steunt op React en dus een groot voordeel oplevert voor ontwikkelaars die reeds vertrouwd zijn met React. Dit is natuurlijk niet het enige voordeel van React Native en ook nadelen zijn aanwezig, net als bij elk framework. Deze sterke en zwakkere punten zijn te vinden in de volgende paragrafen. 

\subsection{Voordelen React Native}
\label{subsec:voordelenReactNative}

Zoals in de inleiding van deze sectie reeds gesteld werd is het feit dat het framework gebaseerd is op React een groot voordeel voor ontwikkelaars die reeds React kennen (de taal en manier van werken is reeds gekend). Ook voor ontwikkelaars die nog geen achtergrond hebben in React levert dit echter een groot voordeel op: ook de 'hot reload' van React werd meegenomen in het framework. Dit houdt in dat een ontwikkelaar aanpassingen kan doen, deze opslaan en direct het resultaat van deze wijzigingen kan zien. Dit in tegenstelling tot native ontwikkeling, waar de build van een applicatie al gauw meerdere minuten in beslag neemt. Voor het op punt stellen van de UI van de applicatie levert de 'hot reload' dus een gevoelige tijdsbesparing op.

React Native heeft natuurlijk ook enkele sterke punten die niks te maken hebben met het feit dat het gebaseerd is op React. Eén daarvan is de bescrhijving van UI componenten met eigen componenten binnen het framework. Dit houdt in dat het framework beschikt over een algemene beschrijving voor een component en bij het effectieve deployen van de applicatie op een bepaald platform wordt deze algemene beschrijving gelinkt aan de specifieke component van dat platform. Het grote voordeel hiervan is dat de ontwikkelaar geen enkele kennis nodig heeft van de native UI componenten. Kennis van de beschrijving van React Native van componenten is voldoende om UI's te maken die er native uitzien. 

Applicaties die er native uitzien ondanks het feit dat ze cross-platform geschreven zijn is één van de grootste voordelen van React Native. Met een eigen definitie van componenten zoals besproken in de vorige paragraaf en directe toegang tot de native API's kunnen ontwikkelaars dezelfde kwaliteitsvolle UI's leveren aan de gebruikers van de applicatie als met een native applicatie. Gezien de overvloed van bestaande applicaties en de eisen die gebruikers stellen aan het uitzicht van een applicatie is dit absoluut noodzakelijk om een applicatie te schrijven die met plezier gebruikt wordt.

Verder is React Native een zeer populair framework (zie figuur \ref{fig:frameworkPopularity}). Het is ontwikkeld en wordt onderhouden door Facebook, maar het framework is open source en alle code ervan is integraal te vinden op GitHub. Door de grote populariteit en het open source karakter van het framework zijn er zeer veel contributors aan het framework en zijn er dus ook regelmatige updates en bug fixes. De contributors van het framework zijn bovendien niet enkel individuen, ook enkele andere bedrijven (zoals o.a. Microsoft) helpen mee met de ontwikkeling van React Native \autocite{Nakazawa2019}. Door de grote en actieve communicty zijn er dus regelmatige updates en bugfixes, maar dat is niet het enige voordeel. Er zijn ook zeer veel externe libraries beschikbaar die veel voorkomende functionaliteiten omvatten in simpele methodes, zodat de ontwikkelaar niet alles zelf hoeft te schrijven. Dit levert een grote tijdswinst op en kan er ook voor zorgen dat er bepaalde functionaliteit toegevoegd kan worden aan de app die de ontwikkelaar zelf niet had kunnen schrijven.

\subsection{Nadelen React Native}
\label{subsec:nadelenReactNative}
