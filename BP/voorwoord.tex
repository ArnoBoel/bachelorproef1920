%%=============================================================================
%% Voorwoord
%%=============================================================================

\chapter*{\IfLanguageName{dutch}{Woord vooraf}{Preface}}
\label{ch:voorwoord}

%% TODO:
%% Het voorwoord is het enige deel van de bachelorproef waar je vanuit je
%% eigen standpunt (``ik-vorm'') mag schrijven. Je kan hier bv. motiveren
%% waarom jij het onderwerp wil bespreken.
%% Vergeet ook niet te bedanken wie je geholpen/gesteund/... heeft

Na lang zoeken naar een onderwerp dat me interesseerde kwam ik de aanbieding van Zoovu tegen. Zij hadden een voorstel om regressie testen van een chatbot te onderzoeken. Chatbots zijn een heel belangrijk deel van vele webshops en dergelijke geworden, die de werklast van werknemers enorm kunnen verminderen. Om er voor te zorgen dat de chatbots die Zoovu oplevert ook voldoen aan de strenge  kwaliteitseisen is testen natuurlijk noodzakelijk. Ik kon hier dus met mijn bachelor proef effectief een bijdrage gaan leveren aan de vooruitgang van een bedrijf, wat we enorm aansprak. Verder was het een volledig nieuw onderwerp (ik had nog geen enkele ervaring met chatbots bij aanvang van mijn bachelorproef) waar ik me helemaal in kon vastbijten. Door de coronacrisis viel het werken in het kantoor van Zoovu helaas in het water, maar ik heb toch mijn best gedaan om er het beste van te maken. 

Een speciale bedanking aan Stijn De Smet voor het opnemen van de rol van co-promotor en om mij te ondersteunen. 