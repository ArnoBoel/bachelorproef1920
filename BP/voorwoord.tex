%%=============================================================================
%% Voorwoord
%%=============================================================================

\chapter*{\IfLanguageName{dutch}{Woord vooraf}{Preface}}
\label{ch:voorwoord}

%% TODO:
%% Het voorwoord is het enige deel van de bachelorproef waar je vanuit je
%% eigen standpunt (``ik-vorm'') mag schrijven. Je kan hier bv. motiveren
%% waarom jij het onderwerp wil bespreken.
%% Vergeet ook niet te bedanken wie je geholpen/gesteund/... heeft

Tijdens het zoeken naar een geschikt onderwerp voor mijn bachelorproef kwam ik reeds snel met het idee om een onderwerp te kiezen in samenwerking met mijn team van mijn stage. Zij kwamen met het idee om een vergelijkende studie te voeren naar cross-platform frameworks, aangezien zij hier reeds mee aan de slag zijn en willen weten welk framework de beste keuze is voor de toekomst. Met mijn bachelorproef kan ik dus een bijdrage leveren om de keuze te vergemakkelijken. Het feit dat mijn bachelorproef een effectief doel heeft is natuurlijk heel erg leuk en goed voor de motivatie. Verder ben ik akkoord gegaan met dit onderwerp omdat het een techologie betreft waar ik zelf nog niet zo heel veel mee in aanraking ben gekomen maar die wel een heel groot potentiëel heeft. Het is dus zeer interessant om me hier in te gaan verdiepen en te ontdekken wat de verschillende mogelijkheden zijn.

Graag wil ik dan ook Sander Nouwynck bedanken voor het onderwerp, het opnemen van het co-promotorschap en de ondersteuning tijdens het effectieve schrijven mijn bachelorproef. Tot slot ook een speciale bedanking aan alle mensen die mijn bachelorproef hebben nagelezen, zonder jullie hadden er enkele pijnlijke schrijffouten blijven staan.