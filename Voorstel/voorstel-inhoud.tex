%---------- Inleiding ---------------------------------------------------------

\section{Introductie} % The \section*{} command stops section numbering
\label{sec:introductie}

In de wereld van vandaag beschikt bijna iedereen over een smartphone. Om deze apparaten ten volle te kunnen benutten zijn er miljoenen applicaties beschikbaar die op hun beurt kunnen rekenen op miljoenen gebruikers. Om aan de steeds groter wordende verwachtingen van de gebruikers te voldoen spenderen ontwikkelaars zeer veel tijd aan het bouwen van nieuwe applicaties en het onderhouden en verbeteren van reeds bestaande applicaties. Op dit moment zijn er twee populaire besturingssystemen waarop deze applicaties moeten kunnen werken voor mobiele gebruikers: iOS en Android. Aangezien ontwikkelaars natuurlijk niet op voorhand reeds een heel groot percentage potentiële gebruikers willen uitsluiten moeten alle applicaties ontwikkeld worden voor beide besturingssystemen. Dit is een tijdrovend en repititief proces, aangezien er in se twee keer hetzelfde gemaakt wordt. Om dit probleem aan te pakken zijn cross-platform frameworks in het leven geroepen. Dit zijn frameworks die de ontwikkelaar in staat stellen om met één enkele broncode hun applicatie te laten werken op verschillende besturingssystemen. Dit levert een grote tijdswinst op voor de ontwikkelaar en is dus een zeer interessante technologie om mee aan de slag te gaan. De laatste jaren zijn meer en meer productenten van frameworks zich hier ook op gaan focussen met als gevolg dat er nu een brede keuze is aan verschillende cross platform frameworks. De vraag dringt zich dus op welke van deze bestaande frameworks de beste keuze is om in de toekomst mee aan de slag te gaan voor het ontwikkelen van mobiele applicaties. In deze studie zullen drie cross platform frameworks van grote producenten met elkaar vergeleken worden: React Native (Facebook), Blazor en MAUI (beiden Microsoft). Concreet zal er gewerkt worden rond de volgende onderzoeksvraag: welk cross platform framework is de beste keuze voor een team van ontwikkelaars om de komende jaren op in te zetten voor het ontwikkelen van mobiele applicaties. Hierbij zal gekeken worden naar de gebruiksvriendelijkheid voor zowel de ontwikkelaar als de gebruiker, de ondersteuning van het framework, de snelheid van apps ontwikkeld met dit framework en de veiligheid van het framework. Van elke technologie zullen de sterke en zwakke punten opgesomd worden, ze zullen onderling vergeleken worden op de hiervoor benoemde punten en in de conclusie zal een aanbeveling gedaan worden welk framework het meest 'future-proof' is. 
%---------- Stand van zaken ---------------------------------------------------

\section{Literatuurstudie}
\label{sec:literatuurstudie}

\subsection{Definitie van regressietesten}

Volgens \textcite{EngstroemRuneson2010} is regressietesten het hertesten van voordien werkende software nadat er wijziging zijn doorgevoerd om te verzekeren dat niet aangepaste software nog steeds werkt als voordien. In een onderzoek naar de definitie van regressietesten in de opinie van experts uit de industrie, kwamen \cite{Minhas2017} tot de conclusie dat de industrie regressietesten op 2 manieren zou beschrijven. Enerzijds is regressietesten een activiteit die vertrouwen geeft over wat er is uitgevoerd en die zekerheid geeft dat er niks is kapot gemaakt. Anderzijds is regressietesten een activiteit die er voor zorgt dat alles correct blijft werken na een wijziging in het systeem en het is een garantie dat het systeem ook in de toekomst zal blijven werken.

\subsection{Activiteiten bij regressietesten}

Volgens  \textcite{AhlamAnsari2016} worden er steeds 3 activiteiten uitgevoerd alvorens de regressietesten effectief te laten runnen:

\begin{itemize}
    \item Selectie van de test case: de relevante test cases voor een specifiek deel van de applicatie of de uitgevoerde aanpassingen.
    \item Minimaliseren van de test case: het aantal test cases wordt beperkt door test cases die hetzelfde testen te verwijderen. 
    \item Prioriteit stellen van bepaalde test cases: de volgorde bepalen waarin de test cases zullen uitgevoerd worden zodat de kans om fouten op te sporen gemaximaliseerd wordt.
\end{itemize}

Deze 3 stappen zijn zeer belangrijk: naarmate een applicatie groter wordt zal ook de test suite (de verzamelig van alle testen) zeer groot worden. Deze kunnen niet allemaal steeds opnieuw uitgevoerd worden (dit zou veel te lang duren en dus ook te veel geld kosten). Er wordt dus telkens slechts een deel van de beschikbare test cases uitgevoerd. De testen moeten echter wel nog steeds zekerheid geven dat de code geen fouten bevat. Het is dus essentieel om een goede selectie van test cases te maken. 

Volgens \textcite{PrashantMalangave2015} is de beste manier om aan 'Test prioritization' te doen kijken naar de code coverage (hoeveel procent van de code effectief getest wordt door deze test). Uit de geselecteerde test cases is de beste strategie om eerst deze met de grootste code coverage te gaan kiezen doordat er meer fouten gevonden kunnen worden als er meer code getest wordt.

\subsection{Combinatorial interaction regression test (CIT)}

Een mogelijke oplossing voor het verkleinen van het aantal test cases dat gebruikt moet worden is volgens \textcite{Qu2014} Combinatorial interaction testing (CIT). Dit is een techniek die gebruik maakt van parameter interacties om fouten te ontdekken, waardoor ook de interactie tussen verschillende compononenten van het systeem behandeld wordt \autocite{Brcic2013}.

% Voor literatuurverwijzingen zijn er twee belangrijke commando's:
% \autocite{KEY} => (Auteur, jBij het aartal) Gebruik dit als de naam van de auteur
%   geen onderdeel is van de zin.
% \textcite{KEY} => Auteur (jaartal)  Gebruik dit als de auteursnaam wel een
%   functie heeft in de zin (bv. ``Uit onderzoek door Doll & Hill (1954) bleek
%   ...'')

%---------- Methodologie ------------------------------------------------------
\section{Methodologie}
\label{sec:methodologie}

Om een goed beeld te krijgen van de reeds bestaande oplossingen voor regressietesten zal er eerst hierover een grondige tudie gevoerd worden. Vervolgens zal er een grondige studie gedaan worden naar de werking van chatbots. Na afloop van beide studies zal er bekeken worden hoe de bestaande oplossingen voor regressietesten geïmplementeerd kunnen worden om regressietesten voor chatbots te gaan ontwikkelen. Om te testen of de aangewezen methodes voor het schrijven van regressietesten effectief bruikbaar zijn bij het testen van chatbots zal er een test case uitgewerkt worden. Aan de hand van de resultaten van deze test case kan er vervolgens bepaald worden of deze methodes effectief een meerwaarde bieden. Voor de gebruiksvriendelijkheid van de methode te testen zal er een klein experiment uitgevoerd worden bij Clever en achteraf een vragenlijst. Op deze manier kan bepaald worden of de voorgestelde methode effectief bruikbaar is en een meerwaarde zal kunnen bieden in de toekomst. 


%---------- Verwachte resultaten ----------------------------------------------
\section{Verwachte resultaten}
\label{sec:verwachte_resultaten}

De lijn van verwachting voor dit onderzoek is dat er een concreet voorstel zal uit voortvloeien om met een bepaalde methode aan de slag te gaan. De kans is echter groot dat er op dit moment nog geen framework beschikbaar is dat helemaal zal voldoen aan de noden van Clever. Gezien het feit dat het ontwikkelen van een test framework niet hoort tot de scope van dit onderzoek zal er in dit geval geen kant en klare oplossing zijn. Wel kan het resultaat van dit onderzoek leiden tot een beter inzicht van hoe er aan de slag kan gegaan worden met de reeds bestaande frameworks.

%---------- Verwachte conclusies ----------------------------------------------
\section{Verwachte conclusies}
\label{sec:verwachte_conclusies}

De verwachte conclusie is dat het schrijven van regressietesten voor chatbots geen sinecure is. Door de enorme hoeveelheid mogelijke input zal nooit de volledige chatbot telkens opnieuw getest kunnen worden. Wat wel mogelijk zal blijken is dat de taalherkenning en de meest voorkomende vragen die de chatbot zal moeten beantwoorden getest zullen kunnen worden. Een aanrading van een framework zal er waarschijnlijk wel inzitten, maar een proof of concept opleveren zal alleen lukken indien er een gratis framework beschikbaar is. Een laatste conclusie zal echter zijn dat het testen van chatbots een tijdsintensieve gebeurtenis zal blijven en dat er nog zeer veel ruimte zal zijn voor verbetering op het vlak van gebruiksvriendelijkheid.

