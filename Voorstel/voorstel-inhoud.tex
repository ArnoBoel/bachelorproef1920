%---------- Inleiding ---------------------------------------------------------

\section{Introductie} % The \section*{} command stops section numbering
\label{sec:introductie}

In de wereld van vandaag beschikt bijna iedereen over een smartphone. Om deze apparaten ten volle te kunnen benutten zijn er miljoenen applicaties beschikbaar die op hun beurt kunnen rekenen op miljoenen gebruikers. Om aan de steeds groter wordende verwachtingen van de gebruikers te voldoen spenderen ontwikkelaars zeer veel tijd aan het bouwen van nieuwe applicaties en het onderhouden en verbeteren van reeds bestaande applicaties. Bij het ontwikkelen van mobiele applicaties komen ontwikkelaars steeds weer dezelfde uitdaging tegen: op dit moment zijn er twee populaire besturingssystemen waarop deze applicaties moeten kunnen werken voor mobiele gebruikers, zijnde iOS en Android (en Windows op de B2B markt). Aangezien ontwikkelaars natuurlijk niet op voorhand reeds een heel groot percentage potentiële gebruikers willen uitsluiten moeten alle applicaties ontwikkeld worden voor beide besturingssystemen. Dit is een tijdrovend en repititief proces, aangezien er in se twee keer hetzelfde gemaakt wordt. Om dit probleem aan te pakken zijn cross platform frameworks in het leven geroepen. Dit zijn frameworks die de ontwikkelaar in staat stellen om met één enkele broncode hun applicatie te laten werken op verschillende besturingssystemen. Dit levert een grote tijdswinst op voor de ontwikkelaar en is dus een zeer interessante technologie om mee aan de slag te gaan. De laatste jaren zijn meer en meer producenten van frameworks zich hier ook op gaan focussen met als gevolg dat er nu een brede keuze is aan verschillende cross platform frameworks. De vraag dringt zich dan ook op welke van deze bestaande frameworks de beste keuze is om in de toekomst mee aan de slag te gaan voor het ontwikkelen van mobiele applicaties. In deze studie zullen drie cross platform frameworks van grote producenten met elkaar vergeleken worden: React Native (Facebook), Flutter (Google) en .NET MAUI (Microsoft). Concreet zal er gewerkt worden rond de volgende onderzoeksvraag: welk cross platform framework is de beste keuze voor een team van ontwikkelaars om de komende jaren op in te zetten voor het ontwikkelen van mobiele applicaties. Hierbij zal gekeken worden naar de gebruiksvriendelijkheid voor zowel de ontwikkelaar als de gebruiker, de ondersteuning van het framework, de snelheid van apps ontwikkeld met dit framework en de mogelijkheid om te communiceren met de native API's. Van elke technologie zullen de sterke en zwakke punten opgesomd worden, ze zullen onderling vergeleken worden op de hiervoor benoemde punten en in de conclusie zal een aanbeveling gedaan worden welk framework het meest 'future-proof' is. 
%---------- Stand van zaken ---------------------------------------------------

\section{Literatuurstudie}
\label{sec:literatuurstudie}

Alvorens er dieper kan ingegaan worden op de specifieke frameworks is het belangrijk om te weten wat een cross platform framework is en wat de voor- en nadelen zijn ten opzichte van een klassiek native framework. Zo wordt ook meteen duidelijk wat al deze frameworks gemeenschappelijk hebben, zodat het vervolgens makkelijker zal zijn om de verschillen te gaan bekijken.

\subsection{Definitie van cross platform frameworks}

Volgens \textcite{El-Kassas2014} en zoals de naam al doet vermoeden is een cross platform framework een framework dat de ontwikkelaar in staat stelt om een applicatie te laten werken op verschillende besturingssystemen. Dit zorgt ervoor dat applicaties op alle besturingssystemen tegelijk beschikbaar zijn en dat de benodigde tijd om dit te doen gevoelig daalt. Net zoals elke technologie zijn er natuurlijk ook bepaalde voor- en nadelen verbonden aan cross platfrom frameworks. Zowel de voordelen als de nadelen zullen in de volgende paragrafen besproken worden.

\subsection{Voordelen van cross platform frameworks}

Er zijn vele voordelen verbonden aan cross platform frameworks in vergelijking met native frameworks, zowel voor de ontwikkelaars als voor de gebruikers van de applicatie. De meeste van deze voordelen komen rechtstreeks voort uit het feit dat er slecht één broncode moet geschreven worden. Ten eerste levert het schrijven van één broncode een enorme tijdswinst op voor de ontwikkelaars. Indien een bepaalde applicatie voor twee verschillende besturingssystemen geschreven moet worden zal ofwel één team er langer over doen of zal er een groter team nodig zijn om allebei de applicaties tegelijk te maken. Door deze beide situaties te voorkomen wordt er een zeer grote kost uitgespaard. 

Een ander voordeel is dat de tijd om de applicatie op de markt te brengen gevoelig daalt. Hierdoor wordt een voordeel op de concurentie behaald die misschien aan een gelijkaardige applicatie aan het werken is. 

Verder heeft dit niet enkel een voordeel op het ontwikkelen van nieuwe applicaties maar ook op het onderhouden en updaten van reeds bestaande applicaties. Bugs moeten slechts één keer opgelost worden en ook nieuwe features hoeven slechts één keer ontwikkeld te worden om op alle besturingssystemen beschikbaar te zijn. Dit leidt tot een groot voordeel voor de gebruikers, die sneller kunnen beschikken over applicaties waarin de foutjes zijn opgelost en waaraan nieuwe features zijn toegevoegd. 

Een enkelvoudige broncode levert niet enkel tijdswinst op, ook de connectie met de cloud en eventuele externe services verloopt eenvoudiger. Door slechts één broncode te hebben is er slechts één plaats waar deze externe services moeten worden ingesteld. Hierdoor is dit makkelijker op te zetten en te onderhouden.

Verder bereikt men door een applicatie te ontwikkelen die werkt op alle besturingssystemen een groter doelpubliek met de applicatie. Niet iedereen heeft nu eenmaal een smartphone die werkt op hetzelfde besturingssysteem. Indien men dus een applicatie zou uitbrengen die enkel werkt op Android of iOS verliest men direct een groot aantal potentiële gebruikers. Ook indien er een nieuw populair besturingssysteem zou bijkomen voor smarpthones levert cross platform een voordeel op: de applicatie zal ook op dit besturingssysteem kunnen werken zonder dat er een nieuwe applicatie geschreven hoeft te worden.

Tot slot heeft het ontwikkelen van één enkele applicatie nog een extra voordeel voor de gebruiker: indien deze overschakelt naar een nieuw besturingssysteem zal hij nog steeds beroep kunnen doen op zijn vertrouwde applicaties. Er is zelfs nog meer: de layout van de applicaties zal identiek het zelfde zijn, aangezien het om één en dezelfde applicatie gaat (mits enkele minieme verschillen eigen aan de verschillende besturingssystemen). Hierdoor moet de gebruiker niet wennen aan een nieuwe layout en stijgt de gebruiksvriendelijkheid en herkenbaarheid.

\subsection{Nadelen van cross platform frameworks}

Natuurlijk zijn er, net zoals bij elke technologie, niet enkel voordelen verbonden aan werken met een cross platform framework. Volgens \textcite{Corral2012} zijn er drie partijen die eventuele nadelen kunnen ondervinden: de gebruikers van de applicatie, de ontwikkelaars van de applicatie en de aanbieders van een platform. Voor deze studie zijn enkel de eerste twee groepen relevant.

Voor de ontwikkelaars is de grootste hindernis de (relatief) beperkte ondersteuning vergeleken met native platforms. Dit komt doordat cross platform een vrij recente technologie is en dus nog niet de grote ervaring heeft die native platforms wel reeds hebben. Een bijkomende uitdaging is dat hoewel de broncode slechts één keer geschreven wordt de applicatie wel op verschillende systemen wordt uitgerold. De applicatie moet dus ook op de verschillende systemen getest en onderhouden worden.

De gebruikers van de applicatie kunnen ook (zij het op een steeds mindere basis) nadelen ondervinden van een applicatie die met een cross platform framework geschreven is. Zo kan het zijn dat een applicatie niet ten volle gebruik maakt van de mogelijkheden van het ene besturingssysteem, omdat het andere besturingssysteem niet over deze mogelijkheden beschikt. Door het feit dat de applicatie geen native applicatie is kan het verder zijn dat de applicatie geen 'native' gevoel geeft: dit wil zeggen dat het er visueel anders kan uitzien dan indien de app geschreven werd met behulp van een native framework. Tot slot kan het zijn dat de applicatie trager is dan native applicaties door een niet optimale code voor dat specifieke besturingssysteem.

\subsection{Achtergrond React Native}

React Native (2015) is een open source cross platform framework dat ontwikkeld is en onderhouden wordt door Facebook. Zoals de naam al doet vermoeden steunt dit framework op React (een Javascript library, ook ontwikkeld door Facebook). Dit is een library die speciaal is ontwikkeld om gebruikersinterfaces te maken. De voordelen van React worden meegenomen in het framework: zo kan er bijvoorbeeld zeer snel vernieuwd worden ('hot reload'). De ontwikkelaar slaat de aanpassingen op en de applicatie geeft deze wijzigingen direct weer, in tegenstelling tot native frameworks waar er steeds moet gewacht worden tot de applicatie 'gebouwd' is. Dit geeft de ontwikkelaar de kans om makkelijker kleine wijzigingen te gaan doorvoeren en het effect van deze wijzigingen sneller te kunnen zien verschijnen.

Een andere eigenschap van React Native is dat het gebruik maakt van eigen componenten om de gebruikersinterface van de applicatie te gaan beschrijven. Deze worden dan gecompileerd naar de corresponderende native componenten. Door op deze manier te werk te gaan kunnen ontwikkelaars gebruik maken van de native API's. Dit levert een groot voordeel op voor de native uitstraling van de applicatie en stelt native ontwikkelaars in staat om snel aan de slag te kunnen met React Native door de grote gelijkenissen met de native omgeving.

Zoals eerder vermeld is React Native ontwikkeld door Facebook. Het is echter een open source framework en de code hiervan is integraal te vinden op GitHub. Er zijn heel veel contributors aan dit framework, waardoor er steeds nieuwe zaken bijkomen die het framework nog beter maken. Dit zijn bovendien niet enkel individuen die hun bijdrage leveren, er zijn ook enkele bedrijven die mee helpen aan de ontwikkeling van React Native.

\subsection{Achtergrond Flutter}

Flutter is een software development kit (SDK) ontwikkeld door Google in 2017.Net zoals React Native is ook Flutter volledig open source. Het kan rekenen op een grote community en voortdurende verdere ontwikkeling. De laatste stabiele versie van Flutter werd uitgebracht door Google op 6/5/2020. Het is dus een zeer recente update waardoor de SDK kan rekenen op de laatste nieuwe ontwikkelingen op het vlak van cross platform development. Flutter apps worden geschreven in Dart, een object georiënteerde programmeertaal gebaseerd op klassen en ontwikkeld door Google. Het grote voordeel van Dart is dat het gecompileerd kan worden naar Javascript maar ook rechtstreeks naar native code, wat een groot voordeel oplevert op het vlak van prestaties. 

Een andere positieve eigenschap is dat de SDK beschikt over zijn eigen widgets. Deze kunnen gebruikt worden om een native gevoel te creëeren voor de applicatie. Er is een library met widgets voor zowel Android als iOS. Door deze widgets te gaan gebruiken hoeft de ontwikkelaar vele zaken niet zelf te gaan definiëren er bekomt men een gebruikersinterface die consistent is met deze van een native applicatie. 

\subsection{Achtergrond .NET MAUI}

.NET MAUI is een cross platform framework, uitgebracht door Microsoft in 2020. Zoals de naam doet vermoeden is het gebaseerd op .NET. Dit betekent dat er geprogrammeerd wordt in C\#, maar er kan ook gebruik gemaakt worden van Blazor. Het framework is nog zeer nieuw en staat nog in zijn kinderschoenen. Microsoft plant om .NET MAUI voor iedereen beschikbaar te maken in november 2021. Net zoals bij React Native wordt er gewerkt met eigen componenten die dan vervolgens worden omgezet naar de overeenkomstige native componenten. .NET MAUI richt zich verder op het volledig beschikbaar maken van de native API's, zodat de ontwikkelaar toegang heeft tot alle mogelijkheden van een specifiek besturingssysteem. 

Verder beschikt .NET MAUI ook over een MVU-patroon (Model-View-Update). Dit houdt in dat er een enkele richting is waarin data kan doorgegeven worden, er wordt een enkele staat bijgehouden van de data voor de gehele applicatie en de gebruikersinterface wordt sneller geüpdate door enkel de noodzakelijke veranderingen door te voeren (hot reload).

% Voor literatuurverwijzingen zijn er twee belangrijke commando's:
% \autocite{KEY} => (Auteur, jBij het aartal) Gebruik dit als de naam van de auteur
%   geen onderdeel is van de zin.
% \textcite{KEY} => Auteur (jaartal)  Gebruik dit als de auteursnaam wel een
%   functie heeft in de zin (bv. ``Uit onderzoek door Doll & Hill (1954) bleek
%   ...'')

%---------- Methodologie ------------------------------------------------------
\section{Methodologie}
\label{sec:methodologie}

Om een goed beeld te krijgen van de eigenschappen van de drie cross platform frameworks die in deze studie aan bod komen wordt er eerst een grondige literatuurstudie uitgevoerd. Hierin wordt nog geen vergelijking gemaakt, enkel feiten over elk framework afzonderlijk komen hier aan bod. Vervolgens zal een vergelijkende studie uitgevoerd worden, waarbij de drie gekozen frameworks vergeleken zullen worden op de volgende punten: 

\begin{itemize}
    \item gebruikservaring van de gebruiker van de applicatie (snelheid, native look \&feel)
    \item performantie van de applicatie
    \item gebruiksvriendelijkheid voor de ontwikkelaar
    \item ondersteuning van de community (updates e.d.)
    \item toegang tot de native libraries \& API's
\end{itemize}

Tijdens het onderzoek zal er ook gefocust worden op de talen waarin applicaties met deze frameworks geschreven worden: React.Js (React Native), Dart (Flutter) en C\# of Blazor (.NET MAUI). Aangezien elk van deze talen zijn eigen sterktes en zwaktes heeft is dit zeker ook een belangrijk punt om mee te gaan onderzoeken. 

Tot slot zal er een enquête gedaan worden binnen het team van delaware naar de reeds bestaande kennis van deze drie frameworks, de voorkeur voor een bepaalde programmeertaal en de haalbaarheid van het overschakelen naar een nieuwe taal. Op deze manier kan er een aanbeveling van een framework gedaan worden specifiek voor het betreffende team.


%---------- Verwachte resultaten ----------------------------------------------
\section{Verwachte resultaten}
\label{sec:verwachte_resultaten}

De lijn van verwachting van dit onderzoek is dat er niet direct een afgetekend verschil zal zijn tussen de drie frameworks. Ze zullen zeker wel verschillen op bepaalde gebieden en elk zal zijn eigen sterktes en zwaktes hebben, maar het zal niet eenduidig duidelijk zijn dat het ene framework beter is dan het andere. Verder wordt er verwacht dat hoe recenter het framework, hoe beter het framework zal scoren op bepaalde vlakken waar het oudere framework het moeilijk mee heeft: de producenten van het recentere framework hebben namelijk de tijd gehad om na te denken over een manier om de tekortkomingen van de andere frameworks te omzeilen.

%---------- Verwachte conclusies ----------------------------------------------
\section{Verwachte conclusies}
\label{sec:verwachte_conclusies}

De conclusie van dit onderzoek zal geen eenduidig antwoord bevatten van welk framework de beste keuze is om in de toekomst mee aan de slag te gaan. Wel zal het enkele fundamentele verschillen blootleggen die het team in staat zullen stellen om een goed doordachte beslissing te nemen. Er zal besloten kunnen worden dat elk framework zijn voor- en nadelen heeft en dat de uiteindelijke keuze voor een groot deel beïnvloed zal worden door de voorkeur en expertise van het team voor een bepaalde taal. Verder zal de keuze ook beïnvloed worden door de soort applicaties die gemaakt moeten worden met het framework. Tot slot zal deze studie uitwijzen dat niet enkel de pure kwaliteiten van het framework op zich maar ook de community en ondersteuning van het framework een zeer belangrijke rol spelen in het beantwoorden van de onderzoeksvragen.

