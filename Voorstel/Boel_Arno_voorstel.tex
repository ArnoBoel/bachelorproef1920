%==============================================================================
% Sjabloon onderzoeksvoorstel bachelorproef
%==============================================================================
% Gebaseerd op LaTeX-sjabloon ‘Stylish Article’ (zie voorstel.cls)
% Auteur: Jens Buysse, Bert Van Vreckem
%
% Compileren in TeXstudio:
%
% - Zorg dat Biber de bibliografie compileert (en niet Biblatex)
%   Options > Configure > Build > Default Bibliography Tool: "txs:///biber"
% - F5 om te compileren en het resultaat te bekijken.
% - Als de bibliografie niet zichtbaar is, probeer dan F5 - F8 - F5
%   Met F8 compileer je de bibliografie apart.
%
% Als je JabRef gebruikt voor het bijhouden van de bibliografie, zorg dan
% dat je in ``biblatex''-modus opslaat: File > Switch to BibLaTeX mode.

\documentclass{voorstel}

\usepackage{lipsum}

%------------------------------------------------------------------------------
% Metadata over het voorstel
%------------------------------------------------------------------------------

%---------- Titel & auteur ----------------------------------------------------

% TODO: geef werktitel van je eigen voorstel op
\PaperTitle{Cross platform frameworks: een vergelijkende studie}
\PaperType{Onderzoeksvoorstel Bachelorproef 2019-2020} % Type document

% TODO: vul je eigen naam in als auteur, geef ook je emailadres mee!
\Authors{Arno Boel\textsuperscript{1}} % Authors
\CoPromotor{Sander Nouwynck\textsuperscript{2} (delaware)}
\affiliation{\textbf{Contact:}
    \textsuperscript{1} \href{mailto:arno.boel@student.hogent.be}{arno.boel@student.hogent.be};
    \textsuperscript{2} \href{mailto:sander.nouwynck@delaware.pro}{sander.nouwynck@delaware.pro};
}

%---------- Abstract ----------------------------------------------------------

\Abstract
{
    In deze studie zal een grondige vergelijking gemaakt worden tussen drie verschillende cross platform frameworks: React Native, Flutter en .NET MAUI. Deze geven een zeer groot voordeel ten opzichte van native frameworks aangezien er slechts één applicatie geschreven hoeft te worden die dan op verschillende platformen kan werken. Het doel van deze studie is om een aanbeveling te kunnen doen welk van deze drie frameworks de beste keuze is om op in te zetten in de toekomst als ontwikkelaar van mobiele applicaties. De verwachte conclusie is dat het meest recente framework qua technische mogelijkheden de beste keuze is, maar dat een framework dat reeds wat langer bestaat een betere keuze is op de korte termijn door het ontbreken van kinderziektes.
}

%---------- Onderzoeksdomein en sleutelwoorden --------------------------------
% TODO: Sleutelwoorden:
%
% Het eerste sleutelwoord beschrijft het onderzoeksdomein. Je kan kiezen uit
% deze lijst:
%
% - Mobiele applicatieontwikkeling
% - Webapplicatieontwikkeling
% - Applicatieontwikkeling (andere)
% - Systeembeheer
% - Netwerkbeheer
% - Mainframe
% - E-business
% - Databanken en big data
% - Machineleertechnieken en kunstmatige intelligentie
% - Andere (specifieer)
%
% De andere sleutelwoorden zijn vrij te kiezen

\Keywords{Mobiele applicatieontwikkeling --- cross platform --- mobile} % Keywords
\newcommand{\keywordname}{Sleutelwoorden} % Defines the keywords heading name

%---------- Titel, inhoud -----------------------------------------------------

\begin{document}

\flushbottom % Makes all text pages the same height
\maketitle % Print the title and abstract box
\tableofcontents % Print the contents section
\thispagestyle{empty} % Removes page numbering from the first page

%------------------------------------------------------------------------------
% Hoofdtekst
%------------------------------------------------------------------------------

% De hoofdtekst van het voorstel zit in een apart bestand, zodat het makkelijk
% kan opgenomen worden in de bijlagen van de bachelorproef zelf.
%---------- Inleiding ---------------------------------------------------------

\section{Introductie} % The \section*{} command stops section numbering
\label{sec:introductie}

Bij het ontwikkelen van een chatbot komen zeer veel moeilijkheden kijken. Eén van deze moeilijkheden is dat een chatbot afhankelijk is van wat een gebruiker ingeeft. Door het feit dat er oneindig veel mogelijke waarden zijn waar de chatbot mee aan de slag moet gaan is het een grote uitdaging om de input van de gebruiker correct te gaan interpreteren en correct te gaan reageren op deze input. Het is dus snel duidelijk dat er zich op beide vlakken vele fouten kunnen voordoen. Om een chatbot te testen moet op dit moment een persoon effectief deze waarden gaan ingeven en gaan kijken wat de reactie van de chatbot is. Deze studie zal zich richten op een onderzoek naar reeds bestaande manieren om regressietesten te gaan schrijven en de ontwikkeling van een test die een goede basis zal vormen waar Clever mee aan de slag kan om hun chatbots verder te gaan testen. Het doel van deze studie is dus duidelijk: een concrete aanbeveling doen van een framework dat in staat is om op een gebruiksvriendelijke manier onderhoudbare regressietesten te gaan schrijven. De onderzoeksvraag voor deze studie luidt als volgt: welk reeds bestaand framework levert de meest onderhoudbare oplossing om op een gebruiksvriendelijke manier regressietesten te schrijven voor een chatbot? Er zijn echter ook nog enkele deelonderzoeksvragen, die ook zeker niet vergeten mogen worden. Een eerste deelvraag is wat de beste methode is om regressietesten op een zo goedkoop mogelijke en zo min mogelijk resource-intensieve manier te gaan uitvoeren. Een volgende vraag is welke van deze methodes bruikbaar zijn om te gaan gebruiken voor het schrijven van regressietesten voor chatbots. 

%---------- Stand van zaken ---------------------------------------------------

\section{Literatuurstudie}
\label{sec:literatuurstudie}

\subsection{Definitie van regressietesten}

Volgens \textcite{EngstroemRuneson2010} is regressietesten het hertesten van voordien werkende software nadat er wijziging zijn doorgevoerd om te verzekeren dat niet aangepaste software nog steeds werkt als voordien. In een onderzoek naar de definitie van regressietesten in de opinie van experts uit de industrie, kwamen \cite{Minhas2017} tot de conclusie dat de industrie regressietesten op 2 manieren zou beschrijven. Enerzijds is regressietesten een activiteit die vertrouwen geeft over wat er is uitgevoerd en die zekerheid geeft dat er niks is kapot gemaakt. Anderzijds is regressietesten een activiteit die er voor zorgt dat alles correct blijft werken na een wijziging in het systeem en het is een garantie dat het systeem ook in de toekomst zal blijven werken.

\subsection{Activiteiten bij regressietesten}

Volgens  \textcite{AhlamAnsari2016} worden er steeds 3 activiteiten uitgevoerd alvorens de regressietesten effectief te laten runnen:

\begin{itemize}
    \item Selectie van de test case: de relevante test cases voor een specifiek deel van de applicatie of de uitgevoerde aanpassingen.
    \item Minimaliseren van de test case: het aantal test cases wordt beperkt door test cases die hetzelfde testen te verwijderen. 
    \item Prioriteit stellen van bepaalde test cases: de volgorde bepalen waarin de test cases zullen uitgevoerd worden zodat de kans om fouten op te sporen gemaximaliseerd wordt.
\end{itemize}

Deze 3 stappen zijn zeer belangrijk: naarmate een applicatie groter wordt zal ook de test suite (de verzamelig van alle testen) zeer groot worden. Deze kunnen niet allemaal steeds opnieuw uitgevoerd worden (dit zou veel te lang duren en dus ook te veel geld kosten). Er wordt dus telkens slechts een deel van de beschikbare test cases uitgevoerd. De testen moeten echter wel nog steeds zekerheid geven dat de code geen fouten bevat. Het is dus essentieel om een goede selectie van test cases te maken. 

Volgens \textcite{PrashantMalangave2015} is de beste manier om aan 'Test prioritization' te doen kijken naar de code coverage (hoeveel procent van de code effectief getest wordt door deze test). Uit de geselecteerde test cases is de beste strategie om eerst deze met de grootste code coverage te gaan kiezen doordat er meer fouten gevonden kunnen worden als er meer code getest wordt.

\subsection{Combinatorial interaction regression test (CIT)}

Een mogelijke oplossing voor het verkleinen van het aantal test cases dat gebruikt moet worden is volgens \textcite{Qu2014} Combinatorial interaction testing (CIT). Dit is een techniek die gebruik maakt van parameter interacties om fouten te ontdekken, waardoor ook de interactie tussen verschillende compononenten van het systeem behandeld wordt \autocite{Brcic2013}.

% Voor literatuurverwijzingen zijn er twee belangrijke commando's:
% \autocite{KEY} => (Auteur, jBij het aartal) Gebruik dit als de naam van de auteur
%   geen onderdeel is van de zin.
% \textcite{KEY} => Auteur (jaartal)  Gebruik dit als de auteursnaam wel een
%   functie heeft in de zin (bv. ``Uit onderzoek door Doll & Hill (1954) bleek
%   ...'')

%---------- Methodologie ------------------------------------------------------
\section{Methodologie}
\label{sec:methodologie}

Om een goed beeld te krijgen van de reeds bestaande oplossingen voor regressietesten zal er eerst hierover een grondige tudie gevoerd worden. Vervolgens zal er een grondige studie gedaan worden naar de werking van chatbots. Na afloop van beide studies zal er bekeken worden hoe de bestaande oplossingen voor regressietesten geïmplementeerd kunnen worden om regressietesten voor chatbots te gaan ontwikkelen. Om te testen of de aangewezen methodes voor het schrijven van regressietesten effectief bruikbaar zijn bij het testen van chatbots zal er een test case uitgewerkt worden. Aan de hand van de resultaten van deze test case kan er vervolgens bepaald worden of deze methodes effectief een meerwaarde bieden. Voor de gebruiksvriendelijkheid van de methode te testen zal er een klein experiment uitgevoerd worden bij Clever en achteraf een vragenlijst. Op deze manier kan bepaald worden of de voorgestelde methode effectief bruikbaar is en een meerwaarde zal kunnen bieden in de toekomst. 


%---------- Verwachte resultaten ----------------------------------------------
\section{Verwachte resultaten}
\label{sec:verwachte_resultaten}

De lijn van verwachting voor dit onderzoek is dat er een concreet voorstel zal uit voortvloeien om met een bepaalde methode aan de slag te gaan. De kans is echter groot dat er op dit moment nog geen framework beschikbaar is dat helemaal zal voldoen aan de noden van Clever. Gezien het feit dat het ontwikkelen van een test framework niet hoort tot de scope van dit onderzoek zal er in dit geval geen kant en klare oplossing zijn. Wel kan het resultaat van dit onderzoek leiden tot een beter inzicht van hoe er aan de slag kan gegaan worden met de reeds bestaande frameworks.

%---------- Verwachte conclusies ----------------------------------------------
\section{Verwachte conclusies}
\label{sec:verwachte_conclusies}

De verwachte conclusie is dat het schrijven van regressietesten voor chatbots geen sinecure is. Door de enorme hoeveelheid mogelijke input zal nooit de volledige chatbot telkens opnieuw getest kunnen worden. Wat wel mogelijk zal blijken is dat de taalherkenning en de meest voorkomende vragen die de chatbot zal moeten beantwoorden getest zullen kunnen worden. Een aanrading van een framework zal er waarschijnlijk wel inzitten, maar een proof of concept opleveren zal alleen lukken indien er een gratis framework beschikbaar is. Een laatste conclusie zal echter zijn dat het testen van chatbots een tijdsintensieve gebeurtenis zal blijven en dat er nog zeer veel ruimte zal zijn voor verbetering op het vlak van gebruiksvriendelijkheid.



%------------------------------------------------------------------------------
% Referentielijst
%------------------------------------------------------------------------------
% TODO: de gerefereerde werken moeten in BibTeX-bestand ``voorstel.bib''
% voorkomen. Gebruik JabRef om je bibliografie bij te houden en vergeet niet
% om compatibiliteit met Biber/BibLaTeX aan te zetten (File > Switch to
% BibLaTeX mode)

\phantomsection
\printbibliography[heading=bibintoc]

\end{document}
